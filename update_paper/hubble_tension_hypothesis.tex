\documentclass[reprint,aps,prd,nofootinbib,floatfix]{revtex4-2}

% ==================== Packages ====================
\usepackage[T1]{fontenc}
\usepackage{lmodern}
\usepackage{amsmath,amssymb,mathtools}
\usepackage{graphicx}
\usepackage{booktabs}
\usepackage[hidelinks]{hyperref}

\begin{document}

\title{Cosmological Implications of the GWTC-3 Modified-Propagation Anomaly: Inference Bias in the Hubble Tension}
\author{Aiden B. Smith}
\affiliation{Independent Researcher}
\date{February 10, 2026}

\begin{abstract}
We investigate cosmological implications of the GWTC-3 O3 modified-propagation anomaly using an updated Planck-facing calibration chain. The upstream anomaly repository is archived on Zenodo (DOI: \href{https://doi.org/10.5281/zenodo.18585598}{10.5281/zenodo.18585598}). A 60-restart Planck+MG global refit defines a revised sound-horizon calibration anchor, $H_0^{\mathrm{Planck,MG}}\!\approx\!68.0$, $\Omega_m^{\mathrm{Planck,MG}}\!\approx\!0.306$, and $A_{\mathrm{lens}}\!\approx\!1.04$ (posterior medians).

The leading result is inference bias: if a modified-gravity truth is analyzed with standard GR sound-horizon compression (standard-ruler inversion), recovered $H_0$ shifts by mean $\Delta H_0=+1.88$ km s$^{-1}$ Mpc$^{-1}$ (fixed $\Omega_m$) or $+4.55$ km s$^{-1}$ Mpc$^{-1}$ (lensing-proxy $\Omega_m$), i.e. typical net displacements of order $+2$ to $+5$ km s$^{-1}$ Mpc$^{-1}$ relative to posterior truth.

By contrast, the direct late-time friction channel alone provides limited closure: after rebasing constrained transfer sweeps, the anchor-relief posterior is $\mathcal{R}_{\mathrm{anchor}}^{\mathrm{GR}}=0.1545$ (mean; p16/p50/p84 $=0.108/0.147/0.189$), about 15\% of the local-versus-Planck baseline gap.

For CMB lensing, baseline CAMB propagation predicts suppressed power at $L\!\sim\!100$ and $L\!\sim\!300$ ($-15.29\%$ and $-9.49\%$ medians). An MG-aware response refit then reaches near-reference quality (median $\chi^2=8.06$ vs Planck-reference 9.04). If that additional response freedom is disallowed, the baseline projection remains in strong tension with Planck lensing.
\end{abstract}
\maketitle

\section{Scope and Framing}
This work treats the O3 modified-propagation signal phenomenologically: given the inferred posterior, what cosmological consequences follow? The O3 anomaly analysis and data products are archived on Zenodo~\cite{O3Zenodo}. We do not re-argue detection significance in this manuscript.

Modified GW propagation has been explored in theory-forward frameworks~\cite{Belgacem2018,Nishizawa2018}. In this follow-up, we assume the running effective Planck mass $M_\star(z)$ associated with GW friction is a universal MG sector ingredient, so the same $M_\star(z)$ trajectory also modifies the background/scalar channels probed by CMB compression and lensing~\cite{BelliniSawicki2014,PogosianSilvestri2016}. Here, we use a data-driven posterior and update the pipeline to answer three questions in one chain:
\begin{enumerate}
\item How much late-time Hubble tension relief remains after recalibrating the sound-horizon calibration anchor?
\item Does Planck 2018 lensing necessarily reject this posterior, or can an MG-aware refit absorb the suppression?
\item How much can GR-based standard-ruler inversion bias inferred $H_0$ if MG truth is assumed?
\end{enumerate}

\section{Pipeline Summary}
Posterior draws are taken from
\texttt{outputs/finalization/highpower\_multistart\_v2/M0\_start101}
and propagated through four linked stages:
\begin{enumerate}
\item \textbf{Global Planck+MG recalibration:} 60-restart multistart fit (\texttt{cpuset 0-59}) to establish updated sound-horizon calibration anchor values.
\item \textbf{Late-time rebasing:} constrained/pilot transfer sweeps are rebased to the updated Planck-like anchor and recompressed into a final relief posterior.
\item \textbf{CMB lensing forecasts:} baseline draw-level CAMB projection to Planck 2018 lensing bandpowers, followed by an MG-aware two-parameter lensing refit.
\item \textbf{Compressed standard-ruler inversion:} GR inversion of $\theta_\star=r_d/D_M(z_\star)$ under fixed-$\Omega_m$ and lensing-proxy-$\Omega_m$ assumptions.
\end{enumerate}
These are targeted forecasts and refits, not a full MG TT/TE/EE perturbation-sector likelihood analysis.

\section{Results}

\subsection{Updated sound-horizon calibration anchor from the global Planck+MG fit}
The 60-restart Planck+MG run completed all restarts with 5 converged minima and 55 max-evaluation exits. Using converged minima only, we obtain:
\begin{equation}
H_0^{\mathrm{Planck,MG}}=68.005302\;(\mathrm{p50}),\quad
\Omega_m^{\mathrm{Planck,MG}}=0.30643039\;(\mathrm{p50}),\quad
A_{\mathrm{lens}}=1.0428117\;(\mathrm{p50}).
\end{equation}

With local reference $H_0^{\mathrm{local}}=73.0$, the baseline gap used in rebased relief calculations is
\begin{equation}
\Delta H_0^{\mathrm{base}}=\left|H_0^{\mathrm{local}}-H_0^{\mathrm{Planck,MG}}\right|=4.994698.
\end{equation}

\subsection{Inference bias from GR standard-ruler inversion}
To isolate model-assumption bias, we treat MG posterior draws as truth and invert
$\theta_\star=r_d/D_M(z_\star)$ with a GR compression model:
\begin{itemize}
\item fixed $\Omega_m=\Omega_m^{\mathrm{Planck,MG}}$: $H_{0,\mathrm{inferred}}$ mean $72.394$ (p50 $73.170$), with mean $\Delta H_0=+1.876$ km s$^{-1}$ Mpc$^{-1}$ relative to draw-level truth;
\item lensing-proxy $\Omega_m$: $H_{0,\mathrm{inferred}}$ mean $75.065$ (p50 $75.226$), with mean $\Delta H_0=+4.547$ km s$^{-1}$ Mpc$^{-1}$.
\end{itemize}
The wider lensing-proxy interval reflects the expected $H_0$--$\Omega_m$ degeneracy once rigid GR compression assumptions are relaxed; in this channel, the analysis releases model-imposed precision rather than exhibiting numerical instability.

Relative to the recalibrated Planck+MG anchor $H_0^{\mathrm{Planck,MG}}=68.005$, the posterior medians shift by:
\begin{equation}
\Delta H_0^{\mathrm{truth}}\approx+2.39,\quad
\Delta H_0^{\mathrm{fixed\ inversion}}\approx+5.16,\quad
\Delta H_0^{\mathrm{lensing\ inversion}}\approx+7.22\;\mathrm{km\ s^{-1}\ Mpc^{-1}}.
\end{equation}

\begin{figure}[htbp]
\centering
\includegraphics[width=0.95\linewidth]{hubble_tension_assets/h0_tension_summary.png}
\caption{$H_0$ tension summary comparing Planck 2018 (GR), direct-friction recalibration, two GR-inversion bias channels, and local-distance-ladder context (SH0ES and TRGB/CCHP~\cite{SH0ES2022,Freedman2019}). The dominant displacement comes from GR standard-ruler inversion bias when MG truth is assumed; the broad lensing-proxy interval is the expected $H_0$--$\Omega_m$ degeneracy after relaxing GR compression priors.}
\label{fig:h0_summary}
\end{figure}

\begin{figure}[htbp]
\centering
\begin{minipage}{0.49\linewidth}
\centering
\includegraphics[width=\linewidth]{hubble_tension_assets/h0_true_vs_inferred_hist_fixed_rebased.png}
\end{minipage}
\hfill
\begin{minipage}{0.49\linewidth}
\centering
\includegraphics[width=\linewidth]{hubble_tension_assets/h0_true_vs_inferred_hist_lensing_rebased.png}
\end{minipage}
\caption{Draw-level $H_0$ truth versus GR-inferred $H_0$ under compressed standard-ruler inversion with fixed-$\Omega_m$ (left) and lensing-proxy-$\Omega_m$ (right). Both assumptions bias inferred $H_0$ upward, with larger displacement in the lensing-proxy case.}
\label{fig:h0_bias}
\end{figure}

\subsection{Direct friction channel after late-time rebasing}
After rebasing constrained transfer sweeps to the updated sound-horizon calibration anchor and applying Monte Carlo calibration:
\begin{equation}
\mathcal{R}_{\mathrm{anchor}}^{\mathrm{GR}} = 0.1545
\quad\text{(mean; p16/p50/p84 }=0.1085/0.1475/0.1891\text{)}.
\end{equation}

Independent robustness and joint-fit diagnostics are:
\begin{itemize}
\item 10-case robustness grid: posterior-shift relief mean $0.5296$ (p50 $0.5125$, p84 $0.5453$), with zero failed cases.
\item Joint SN+BAO+CC transfer fit: relief posterior mean $0.8329$ (p50 $0.8386$), but
\begin{equation}
\log\mathrm{BF}_{\mathrm{transfer/no\text{-}transfer}}=-0.533,
\end{equation}
so explicit transfer terms are not favored in this setup.
\end{itemize}
The high-$z$ transfer-bias sensitivity map used for calibration has been moved to supplemental material (Fig.~S1).

\subsection{CMB lensing: baseline suppression and MG-aware response freedom}
Baseline draw-level CAMB projection against Planck 2018 lensing bandpowers (\texttt{consext8}, 64 draws) gives:
\begin{equation}
\left.\frac{C_L^{\phi\phi}(\mathrm{MG})}{C_L^{\phi\phi}(\mathrm{Planck\ ref})}\right|_{L\approx106}
=0.847^{+0.091}_{-0.127},
\end{equation}
\begin{equation}
\left.\frac{C_L^{\phi\phi}(\mathrm{MG})}{C_L^{\phi\phi}(\mathrm{Planck\ ref})}\right|_{L\approx286}
=0.905^{+0.068}_{-0.080},
\end{equation}
with median suppressions of $-15.29\%$ and $-9.49\%$. The baseline fit quality is poor relative to the Planck-reference model:
\begin{equation}
\chi^2_{\mathrm{MG,baseline}}\;(\mathrm{median})=51.77,\qquad
\chi^2_{\mathrm{Planck\ ref}}=9.04,
\end{equation}
and only $3.1\%$ of draws outperform the reference. A 32-draw cross-check from an independent posterior sample is more discrepant ($-18.66\%$ at $L\approx106$, $-11.29\%$ at $L\approx286$; $p_{\mathrm{better}}=0$).

To test whether this baseline mismatch is rigid, we perform an MG-aware lensing refit (32 draws) with a phenomenological effective-$M_\star^2$ amplitude plus $\ell$-tilt response. This freedom is motivated by scalar-tensor/EFT treatments where matter-growth and light-deflection responses need not track identically and can acquire scale dependence~\cite{BelliniSawicki2014,PogosianSilvestri2016}. The refit removes the baseline mismatch:
\begin{equation}
\chi^2_{\mathrm{MG\ refit}}\;(\mathrm{median})=8.06,
\end{equation}
better than the Planck-reference $\chi^2=9.04$ in 100\% of refit draws. The fitted median response corresponds to
\begin{equation}
\frac{M_\star^2(z=0)}{M_\star^2(z\gg1)}\simeq0.901
\end{equation}
(about a 9.9\% drop), with small residual suppression at $L\approx286$.

\begin{figure}[htbp]
\centering
\includegraphics[width=0.95\linewidth]{hubble_tension_assets/clpp_overlay_baseline_vs_mgfit.png}
\caption{Planck 2018 lensing bandpowers with baseline MG projection and MG-aware refit overlay. The refit absorbs the baseline suppression and restores near-reference fit quality.}
\label{fig:clpp_refit}
\end{figure}

\section{Discussion and Conclusion}
The main implication is that inference bias from GR-assumed standard-ruler inversion can be cosmologically large if the O3 modified-propagation posterior corresponds to physical MG truth. In this pipeline, that channel displaces recovered $H_0$ by order $+2$ to $+5$ km s$^{-1}$ Mpc$^{-1}$ relative to draw-level truth, with larger shifts relative to the recalibrated Planck+MG anchor.

The direct friction channel remains subdominant in the constrained rebased analysis: $\mathcal{R}_{\mathrm{anchor}}^{\mathrm{GR}}\simeq0.15$. This means the principal lever in this study is model-assumption bias from GR-assumed standard-ruler inversion, not direct late-time closure alone.

For CMB lensing, baseline propagation is strongly discrepant with Planck 2018. An MG-aware response refit motivated by effective-coupling freedom in scalar-tensor/EFT descriptions restores near-reference likelihood performance. If that response freedom is not permitted, the baseline projection remains in strong tension with lensing data.

Taken together, these results recast the follow-up question from ``does friction alone close the full tension?'' to ``how much of the inferred early-versus-late mismatch can come from GR-compression bias when MG truth is present?'' In this analysis, that range is material and should be included in future MG-aware CMB-to-late-time consistency tests.

\clearpage

\section*{Reproducibility}
Core scripts used in this follow-up are:
\begin{itemize}
\item \texttt{scripts/run\_planck\_global\_mg\_refit\_multistart.py}
\item \texttt{scripts/rebase\_bias\_transfer\_sweep\_to\_planck\_ref.py}
\item \texttt{scripts/run\_hubble\_tension\_final\_relief\_posterior.py}
\item \texttt{scripts/run\_hubble\_tension\_mg\_forecast\_robustness\_grid.py}
\item \texttt{scripts/run\_joint\_transfer\_bias\_fit.py}
\item \texttt{scripts/run\_hubble\_tension\_cmb\_forecast.py}
\item \texttt{scripts/run\_hubble\_tension\_mg\_lensing\_refit.py}
\item \texttt{scripts/run\_hubble\_tension\_early\_universe\_bias.py}
\end{itemize}

\acknowledgments
This work relied extensively on A.I.-assisted tools for code development, pipeline execution support, figure generation, and manuscript drafting/editing.

\section*{Data Availability and DOIs}
The follow-up uses posterior products from the O3 anomaly pipeline and public cosmology datasets. Data provenance and DOIs are:
\begin{itemize}
\item O3 modified-gravity tension anomaly repository (Zenodo): DOI \href{https://doi.org/10.5281/zenodo.18585598}{10.5281/zenodo.18585598}.
\item O3 search-sensitivity injection data used in upstream calibration (Zenodo): DOI \href{https://doi.org/10.5281/zenodo.7890437}{10.5281/zenodo.7890437}.
\item GWTC-3 catalog paper: DOI \href{https://doi.org/10.1103/PhysRevX.13.041039}{10.1103/PhysRevX.13.041039}.
\item Pantheon+ cosmology constraints: DOI \href{https://doi.org/10.3847/1538-4357/ac8e04}{10.3847/1538-4357/ac8e04}.
\item SH0ES local-$H_0$ reference: DOI \href{https://doi.org/10.3847/2041-8213/ac5c5b}{10.3847/2041-8213/ac5c5b}.
\item TRGB/CCHP local-$H_0$ context reference: DOI \href{https://doi.org/10.3847/1538-4357/ab2f73}{10.3847/1538-4357/ab2f73}.
\item SDSS DR12 BOSS consensus BAO (source of \texttt{sdss\_DR12Consensus\_bao.dat}): DOI \href{https://doi.org/10.1093/mnras/stx721}{10.1093/mnras/stx721}.
\item eBOSS DR16 cosmological compilation (source class for \texttt{sdss\_DR16\_LRG\_BAO\_DMDH.dat}): DOI \href{https://doi.org/10.1103/PhysRevD.103.083533}{10.1103/PhysRevD.103.083533}.
\item DESI 2024 BAO cosmological constraints (source class for \texttt{desi\_2024\_gaussian\_bao\_ALL\_GCcomb\_mean.txt}): DOI \href{https://doi.org/10.1088/1475-7516/2025/02/021}{10.1088/1475-7516/2025/02/021}.
\item Cosmic-chronometer compilation components used in \texttt{Hz\_BC03\_all.dat}: DOIs \href{https://doi.org/10.1088/1475-7516/2012/08/006}{10.1088/1475-7516/2012/08/006}, \href{https://doi.org/10.1103/PhysRevD.71.123001}{10.1103/PhysRevD.71.123001}, and \href{https://doi.org/10.1088/1475-7516/2010/02/008}{10.1088/1475-7516/2010/02/008}.
\item Planck 2018 cosmological-parameter and lensing references: DOIs \href{https://doi.org/10.1051/0004-6361/201833910}{10.1051/0004-6361/201833910} and \href{https://doi.org/10.1051/0004-6361/201833886}{10.1051/0004-6361/201833886}.
\end{itemize}

\begin{thebibliography}{99}

\bibitem{O3Zenodo}
A.~B.~Smith,
``O3 Modified Gravity Tension Replication,''
Zenodo (2026), DOI: \href{https://doi.org/10.5281/zenodo.18585598}{10.5281/zenodo.18585598}.

\bibitem{O3InjZenodo}
LIGO Scientific Collaboration, Virgo Collaboration, and KAGRA Collaboration,
``GWTC-3: Compact Binary Coalescences Observed by LIGO and Virgo During the Second Part of the Third Observing Run --- O3 search sensitivity estimates,''
Zenodo (2023), DOI: \href{https://doi.org/10.5281/zenodo.7890437}{10.5281/zenodo.7890437}.

\bibitem{GWTC3}
R.~Abbott \textit{et al.} (LIGO Scientific Collaboration, Virgo Collaboration, and KAGRA Collaboration),
``GWTC-3: Compact Binary Coalescences Observed by LIGO and Virgo During the Second Part of the Third Observing Run,''
\emph{Phys.\ Rev.\ X} \textbf{13}, 041039 (2023),
DOI: \href{https://doi.org/10.1103/PhysRevX.13.041039}{10.1103/PhysRevX.13.041039}.

\bibitem{PantheonPlus}
D.~Brout \textit{et al.},
``The Pantheon+ Analysis: Cosmological Constraints,''
\emph{Astrophys.\ J.} \textbf{938}, 110 (2022),
DOI: \href{https://doi.org/10.3847/1538-4357/ac8e04}{10.3847/1538-4357/ac8e04}.

\bibitem{SH0ES2022}
A.~G.~Riess \textit{et al.},
``A Comprehensive Measurement of the Local Value of the Hubble Constant with 1 km s$^{-1}$ Mpc$^{-1}$ Uncertainty from the Hubble Space Telescope and the SH0ES Team,''
\emph{Astrophys.\ J.\ Lett.} \textbf{934}, L7 (2022),
DOI: \href{https://doi.org/10.3847/2041-8213/ac5c5b}{10.3847/2041-8213/ac5c5b}.

\bibitem{Freedman2019}
W.~L.~Freedman \textit{et al.},
``The Carnegie-Chicago Hubble Program. VIII. An independent determination of the Hubble constant based on the tip of the red giant branch,''
\emph{Astrophys.\ J.} \textbf{882}, 34 (2019),
DOI: \href{https://doi.org/10.3847/1538-4357/ab2f73}{10.3847/1538-4357/ab2f73}.

\bibitem{BOSSDR12}
S.~Alam \textit{et al.},
``The clustering of galaxies in the completed SDSS-III Baryon Oscillation Spectroscopic Survey: cosmological analysis of the DR12 galaxy sample,''
\emph{Mon.\ Not.\ R.\ Astron.\ Soc.} \textbf{470}, 2617 (2017),
DOI: \href{https://doi.org/10.1093/mnras/stx721}{10.1093/mnras/stx721}.

\bibitem{eBOSSDR16}
S.~Alam \textit{et al.},
``Completed SDSS-IV extended Baryon Oscillation Spectroscopic Survey: Cosmological implications from two decades of spectroscopic surveys at the Apache Point Observatory,''
\emph{Phys.\ Rev.\ D} \textbf{103}, 083533 (2021),
DOI: \href{https://doi.org/10.1103/PhysRevD.103.083533}{10.1103/PhysRevD.103.083533}.

\bibitem{DESI2024VI}
DESI Collaboration,
``DESI 2024 VI: cosmological constraints from the measurements of baryon acoustic oscillations,''
\emph{J.\ Cosmol.\ Astropart.\ Phys.} \textbf{02} (2025) 021,
DOI: \href{https://doi.org/10.1088/1475-7516/2025/02/021}{10.1088/1475-7516/2025/02/021}.

\bibitem{Moresco2012}
M.~Moresco \textit{et al.},
``Improved constraints on the expansion rate of the Universe up to $z\sim1.1$ from the spectroscopic evolution of cosmic chronometers,''
\emph{J.\ Cosmol.\ Astropart.\ Phys.} \textbf{08} (2012) 006,
DOI: \href{https://doi.org/10.1088/1475-7516/2012/08/006}{10.1088/1475-7516/2012/08/006}.

\bibitem{Simon2005}
J.~Simon, L.~Verde, and R.~Jimenez,
``Constraints on the redshift dependence of the dark energy potential,''
\emph{Phys.\ Rev.\ D} \textbf{71}, 123001 (2005),
DOI: \href{https://doi.org/10.1103/PhysRevD.71.123001}{10.1103/PhysRevD.71.123001}.

\bibitem{Stern2010}
D.~Stern \textit{et al.},
``Cosmic chronometers: constraining the equation of state of dark energy. I: $H(z)$ measurements,''
\emph{J.\ Cosmol.\ Astropart.\ Phys.} \textbf{02} (2010) 008,
DOI: \href{https://doi.org/10.1088/1475-7516/2010/02/008}{10.1088/1475-7516/2010/02/008}.

\bibitem{Planck2018VI}
N.~Aghanim \textit{et al.} (Planck Collaboration),
``Planck 2018 results. VI. Cosmological parameters,''
\emph{Astron.\ Astrophys.} \textbf{641}, A6 (2020),
DOI: \href{https://doi.org/10.1051/0004-6361/201833910}{10.1051/0004-6361/201833910}.

\bibitem{Planck2018VIII}
N.~Aghanim \textit{et al.} (Planck Collaboration),
``Planck 2018 results. VIII. Gravitational lensing,''
\emph{Astron.\ Astrophys.} \textbf{641}, A8 (2020),
DOI: \href{https://doi.org/10.1051/0004-6361/201833886}{10.1051/0004-6361/201833886}.

\bibitem{Belgacem2018}
E.~Belgacem, Y.~Dirian, S.~Foffa, and M.~Maggiore,
``Modified gravitational-wave propagation and standard sirens,''
\emph{Phys.\ Rev.\ D} \textbf{98}, 023510 (2018),
DOI: \href{https://doi.org/10.1103/PhysRevD.98.023510}{10.1103/PhysRevD.98.023510}.

\bibitem{Nishizawa2018}
A.~Nishizawa,
``Generalized framework for testing gravity with gravitational-wave propagation,''
\emph{Phys.\ Rev.\ D} \textbf{97}, 104037 (2018),
DOI: \href{https://doi.org/10.1103/PhysRevD.97.104037}{10.1103/PhysRevD.97.104037}.

\bibitem{BelliniSawicki2014}
E.~Bellini and I.~Sawicki,
``Maximal freedom at minimum cost: linear large-scale structure in general modifications of gravity,''
\emph{J.\ Cosmol.\ Astropart.\ Phys.} \textbf{07} (2014) 050,
DOI: \href{https://doi.org/10.1088/1475-7516/2014/07/050}{10.1088/1475-7516/2014/07/050}.

\bibitem{PogosianSilvestri2016}
L.~Pogosian and A.~Silvestri,
``What can cosmology tell us about gravity? Constraining Horndeski gravity with $\Sigma$ and $\mu$,''
\emph{Phys.\ Rev.\ D} \textbf{94}, 104014 (2016),
DOI: \href{https://doi.org/10.1103/PhysRevD.94.104014}{10.1103/PhysRevD.94.104014}.

\end{thebibliography}

\end{document}
