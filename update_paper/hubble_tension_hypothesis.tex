\documentclass[11pt]{article}

% ==================== Packages ====================
\usepackage[T1]{fontenc}
\usepackage{lmodern}
\usepackage[margin=1in]{geometry}
\usepackage{amsmath,amssymb}
\usepackage{mathtools}
\usepackage{booktabs}
\usepackage{enumitem}
\usepackage{microtype}
\usepackage{graphicx}
\usepackage{array}
\usepackage{tabularx}
\usepackage[hidelinks]{hyperref}
\usepackage{lineno}

% --- graphics ---
\graphicspath{{./}{update_paper/}{hubble_tension_assets/}{update_paper/hubble_tension_assets/}}

% --- robust figure include ---
\newcommand{\maybeincludegraphics}[2][]{%
  \IfFileExists{#2}{%
    \includegraphics[#1]{#2}%
  }{%
    \IfFileExists{hubble_tension_assets/#2}{%
      \includegraphics[#1]{hubble_tension_assets/#2}%
    }{%
      \IfFileExists{update_paper/hubble_tension_assets/#2}{%
        \includegraphics[#1]{update_paper/hubble_tension_assets/#2}%
      }{%
        \fbox{\parbox{0.9\linewidth}{\centering\small Missing figure: \texttt{\detokenize{#2}}}}%
      }%
    }%
  }%
}

% ==================== Macros ====================
\newcommand{\LPD}{\mathrm{LPD}}
\newcommand{\dL}{d_L}

\title{Hypothesis-Conditioned Forecast of Hubble-Tension Relief\\
\large Assuming the GWTC-3 Dark-Siren Propagation Signal is Physical}
\author{Aiden B. Smith}
\date{February 10, 2026}

\begin{document}
\linenumbers
\maketitle

\begin{abstract}
This work presents hypothesis-conditioned forecasts for the Hubble tension under the assumption that the O3 dark-siren modified-propagation anomaly is physical. The upstream anomaly analysis, archived on Zenodo (DOI: \href{https://doi.org/10.5281/zenodo.18584705}{10.5281/zenodo.18584705}), reports $\Delta\LPD_{\mathrm{tot}}\simeq+3.67$, whereas a GR-truth injection calibration gives mean $-0.839$, standard deviation $0.240$, and maximum $+0.076$ over 512 realizations. Posterior draws from the reconstructed modified-gravity model are propagated into late-time anchor observables, CMB-lensing forecasts, and compressed early-universe inversions.

In the constrained, repeatability-calibrated endpoint, the preferred anchor-based relief posterior is moderate: $\mathcal{R}_{\mathrm{anchor}}^{\mathrm{GR}}$ has mean $0.246$ with p16/p50/p84 $=0.205/0.239/0.277$, and the local-versus-high-$z$ GR gap is typically $\sim1.17\sigma$. A joint SN+BAO+CC transfer-bias fit with O3 metadata yields $\log\mathrm{BF}_{\mathrm{transfer/no-transfer}}=-0.533$, indicating no preference for explicit transfer terms in this setup.

For CMB signatures, CAMB-based propagation to Planck 2018 lensing bandpowers predicts suppressed lensing power, with median shifts of about $-14.9\%$ near $L\simeq100$ and $-8.4\%$ near $L\simeq300$ in a direct 16-draw pilot. Compressed $\theta_\star$ inversion under GR assumptions raises inferred $H_0$ relative to model truth: median inferred $H_0\simeq72.55$ (fixed $\Omega_m$) or $75.34$ (lensing-proxy $\Omega_m$). Under the physical-signal hypothesis, partial tension relief is plausible, but decisive resolution requires full Boltzmann-level modified-gravity refits and independent siren data.
\end{abstract}

\section{Motivation and Scope}
The present analysis is conditional: it does not re-establish the O3 anomaly detection claim, but asks what follows if that signal is physical. The O3 anomaly replication repository is public on Zenodo (DOI: \href{https://doi.org/10.5281/zenodo.18584705}{10.5281/zenodo.18584705}) and provides the calibrated baseline used here. The follow-up objective is to convert that hypothesis into quantitative predictions for:
\begin{enumerate}[leftmargin=2em]
\item late-time inferred-$H_0$ behavior,
\item transfer-bias robustness across SN+BAO+CC,
\item CMB-lensing and compressed early-universe inference shifts under GR interpretation.
\end{enumerate}

\section{Forecast Definitions}
Posterior draws are taken from
\texttt{outputs/finalization/highpower\_multistart\_v2/M0\_start101}
and propagated through synthetic anchor and CMB inference pipelines.

\subsection{Late-time relief metrics}
Define the baseline local-versus-Planck gap
\begin{equation}
\Delta H_0^{\mathrm{base}} \equiv \left|H_0^{\mathrm{local}}-H_0^{\mathrm{Planck}}\right|,
\end{equation}
and the posterior-gap relief fraction
\begin{equation}
\mathcal{R}_{\mathrm{post}} \equiv 1 -
\frac{\left|H_{0,\mathrm{MG}}^{\mathrm{p50}}-H_0^{\mathrm{local}}\right|}{\Delta H_0^{\mathrm{base}}}.
\end{equation}

The preferred estimator is anchor-based. For each anchor redshift $z_a$, a synthetic $H(z_a)$ is generated under model truth and inverted with GR assumptions:
\begin{equation}
H_{0,\mathrm{GR}}(z_a) = \frac{H_{\mathrm{obs}}(z_a)}
{\sqrt{\Omega_{m0}^{\mathrm{GR}}(1+z_a)^3 + (1-\Omega_{m0}^{\mathrm{GR}})}}.
\end{equation}
The anchor-averaged relief statistic is
\begin{equation}
\mathcal{R}_{\mathrm{anchor}}^{\mathrm{GR}} \equiv 1 -
\frac{\left| \overline{H_{0,\mathrm{GR}}} - H_0^{\mathrm{local}} \right|}{\Delta H_0^{\mathrm{base}}}.
\end{equation}

\subsection{CMB-focused tests}
Two CMB-oriented tests are used:
\begin{enumerate}[leftmargin=2em]
\item draw-level propagation of $(H_0,\Omega_{m0},\Omega_{k0},\sigma_8)$ to Planck 2018 lensing bandpowers (template-proxy and direct CAMB modes),
\item compressed early-universe inversion using $\theta_\star=r_d/D_M(z_\star)$ under GR assumptions, with alternative assumptions for inferred $\Omega_m$.
\end{enumerate}
These tests target inference shifts and predicted signatures; they are not full TT/TE/EE Boltzmann likelihood refits.

\section{Results}

\subsection{Late-time forecast and robustness}
Using $z_a=\{0.2,0.35,0.5,0.62\}$, 20,000 Monte Carlo replicates per anchor, and reference values $H_0^{\mathrm{local}}=73.0\pm1.0$ and $H_0^{\mathrm{Planck}}=67.4\pm0.5$:
\begin{itemize}[leftmargin=2em]
\item model-truth posterior gives $H_0^{\mathrm{p50}}\simeq70.39$ (p16/p84 $=67.70/73.39$),
\item single-run posterior-gap relief is $\mathcal{R}_{\mathrm{post}}\simeq0.534$,
\item constrained endpoint gives $\mathcal{R}_{\mathrm{anchor}}^{\mathrm{GR}}$ mean $0.246$ with p16/p50/p84 $=0.205/0.239/0.277$,
\item typical local-versus-high-$z$ GR gap significance is $\sim1.17\sigma$.
\end{itemize}

The joint transfer-bias fit over SN+BAO+CC (with O3 support as metadata) yields
\begin{equation}
\log\mathrm{BF}_{\mathrm{transfer/no\text{-}transfer}} = -0.533,
\end{equation}
so explicit transfer terms are not preferred by these data in this configuration.

\begin{figure}[h!]
\centering
\maybeincludegraphics[width=0.92\linewidth]{h_ratio_vs_planck.png}
\caption{Forecasted expansion-ratio envelope under model truth:
$H_{\mathrm{MG}}(z)/H_{\Lambda\mathrm{CDM,Planck}}(z)$.}
\label{fig:hz_ratio}
\end{figure}

\begin{figure}[h!]
\centering
\maybeincludegraphics[width=0.92\linewidth]{relief_vs_highz_bias.png}
\caption{Anchor-based relief sensitivity to injected high-$z$ calibration bias (pilot and constrained sweeps).}
\label{fig:relief_bias}
\end{figure}

\subsection{CMB lensing signature forecast}
The direct CAMB pilot run (16 posterior draws, Planck 2018 lensing bandpowers) finds median suppression of the lensing spectrum relative to the Planck-reference model:
\begin{equation}
\left.\frac{C_L^{\phi\phi}(\mathrm{MG})}{C_L^{\phi\phi}(\mathrm{Planck\ ref})}\right|_{L\approx100}\simeq0.851,
\qquad
\left.\frac{C_L^{\phi\phi}(\mathrm{MG})}{C_L^{\phi\phi}(\mathrm{Planck\ ref})}\right|_{L\approx300}\simeq0.916.
\end{equation}
In the same pilot, only $12.5\%$ of draws outperform the Planck-reference model in lensing-bandpower $\chi^2$.

\begin{figure}[h!]
\centering
\maybeincludegraphics[width=0.92\linewidth]{clpp_ratio_to_ref.png}
\caption{Predicted CMB lensing ratio $C_L^{\phi\phi}(\mathrm{MG})/C_L^{\phi\phi}(\mathrm{Planck\ ref})$ from the direct CAMB pilot.}
\label{fig:clpp_ratio}
\end{figure}

\subsection{Early-universe GR mis-inference test}
Compressed $\theta_\star$ inversion under GR assumptions gives systematic upward shifts in inferred $H_0$ relative to model-truth draws:
\begin{itemize}[leftmargin=2em]
\item fixed-Planck $\Omega_m$ assumption: inferred $H_0$ mean $71.72$, p50 $72.55$, mean $\Delta H_0\equiv H_0^{\mathrm{inf}}-H_0^{\mathrm{true}}\simeq+1.14$ km s$^{-1}$ Mpc$^{-1}$,
\item lensing-proxy $\Omega_m$ assumption: inferred $H_0$ mean $75.15$, p50 $75.34$, mean $\Delta H_0\simeq+4.57$ km s$^{-1}$ Mpc$^{-1}$.
\end{itemize}

The implied sound-horizon shift required to force exact local-$H_0$ matching is modest in central tendency but broad in distribution: mean $\Delta r_d/r_d\approx-1.74\%$ (fixed-$\Omega_m$ mode) or $+2.92\%$ (lensing-proxy mode).

\begin{figure}[h!]
\centering
\maybeincludegraphics[width=0.48\linewidth]{h0_true_vs_inferred_hist_fixed.png}\hfill
\maybeincludegraphics[width=0.48\linewidth]{h0_true_vs_inferred_hist_lensing.png}
\caption{Histogram-level comparison of true and GR-inferred $H_0$ in compressed early-universe inversion tests. Left: fixed-$\Omega_m$ assumption. Right: lensing-proxy-$\Omega_m$ assumption.}
\label{fig:h0_early_hist}
\end{figure}

\section{Interpretation}
Under the physical-signal hypothesis, the model predicts upward pressure on GR-inferred early/high-$z$ $H_0$ and non-negligible local-versus-high-$z$ tension relief, but not a standalone full resolution. The late-time constrained endpoint remains moderate, and CMB-facing tests indicate detectable but model-dependent signatures that require full perturbation-level treatment for definitive statements.

\section*{Reproducibility}
Core scripts used in this follow-up:
\begin{itemize}[leftmargin=2em]
\item \texttt{scripts/run\_hubble\_tension\_mg\_forecast.py}
\item \texttt{scripts/run\_hubble\_tension\_mg\_forecast\_robustness\_grid.py}
\item \texttt{scripts/run\_hubble\_tension\_bias\_transfer\_sweep.py}
\item \texttt{scripts/run\_hubble\_tension\_final\_relief\_posterior.py}
\item \texttt{scripts/run\_joint\_transfer\_bias\_fit.py}
\item \texttt{scripts/run\_hubble\_tension\_cmb\_forecast.py}
\item \texttt{scripts/run\_hubble\_tension\_early\_universe\_bias.py}
\end{itemize}

\section*{Data Availability and DOIs}
The follow-up uses posterior products from the O3 anomaly pipeline and public cosmology datasets. Data provenance and DOIs are:
\begin{itemize}[leftmargin=2em]
\item O3 modified-gravity tension replication repository (Zenodo): DOI \href{https://doi.org/10.5281/zenodo.18584705}{10.5281/zenodo.18584705}.
\item O3 search-sensitivity injection data used in upstream calibration (Zenodo): DOI \href{https://doi.org/10.5281/zenodo.7890437}{10.5281/zenodo.7890437}.
\item GWTC-3 catalog paper: DOI \href{https://doi.org/10.1103/PhysRevX.13.041039}{10.1103/PhysRevX.13.041039}.
\item Pantheon+ cosmology constraints: DOI \href{https://doi.org/10.3847/1538-4357/ac8e04}{10.3847/1538-4357/ac8e04}.
\item SH0ES local-$H_0$ reference: DOI \href{https://doi.org/10.3847/2041-8213/ac5c5b}{10.3847/2041-8213/ac5c5b}.
\item SDSS DR12 BOSS consensus BAO (source of \path{sdss_DR12Consensus_bao.dat}): DOI \href{https://doi.org/10.1093/mnras/stx721}{10.1093/mnras/stx721}.
\item eBOSS DR16 cosmological compilation (source class for \path{sdss_DR16_LRG_BAO_DMDH.dat}): DOI \href{https://doi.org/10.1103/PhysRevD.103.083533}{10.1103/PhysRevD.103.083533}.
\item DESI 2024 BAO cosmological constraints (source class for \path{desi_2024_gaussian_bao_ALL_GCcomb_mean.txt}): DOI \href{https://doi.org/10.1088/1475-7516/2025/02/021}{10.1088/1475-7516/2025/02/021}.
\item Cosmic-chronometer compilation components used in \path{Hz_BC03_all.dat}: DOIs \href{https://doi.org/10.1088/1475-7516/2012/08/006}{10.1088/1475-7516/2012/08/006}, \href{https://doi.org/10.1103/PhysRevD.71.123001}{10.1103/PhysRevD.71.123001}, and \href{https://doi.org/10.1088/1475-7516/2010/02/008}{10.1088/1475-7516/2010/02/008}.
\item Planck 2018 cosmological parameters and lensing references: DOIs \href{https://doi.org/10.1051/0004-6361/201833910}{10.1051/0004-6361/201833910} and \href{https://doi.org/10.1051/0004-6361/201833886}{10.1051/0004-6361/201833886}.
\end{itemize}

\begin{thebibliography}{99}

\bibitem{O3Zenodo}
A.~B.~Smith,
``O3 Modified Gravity Tension Replication,''
Zenodo (2026), DOI: \href{https://doi.org/10.5281/zenodo.18584705}{10.5281/zenodo.18584705}.

\bibitem{O3InjZenodo}
LIGO Scientific Collaboration, Virgo Collaboration, and KAGRA Collaboration,
``GWTC-3: Compact Binary Coalescences Observed by LIGO and Virgo During the Second Part of the Third Observing Run --- O3 search sensitivity estimates,''
Zenodo (2023), DOI: \href{https://doi.org/10.5281/zenodo.7890437}{10.5281/zenodo.7890437}.

\bibitem{GWTC3}
R.~Abbott \textit{et al.} (LIGO Scientific Collaboration, Virgo Collaboration, and KAGRA Collaboration),
``GWTC-3: Compact Binary Coalescences Observed by LIGO and Virgo During the Second Part of the Third Observing Run,''
\emph{Phys.\ Rev.\ X} \textbf{13}, 041039 (2023),
DOI: \href{https://doi.org/10.1103/PhysRevX.13.041039}{10.1103/PhysRevX.13.041039}.

\bibitem{PantheonPlus}
D.~Brout \textit{et al.},
``The Pantheon+ Analysis: Cosmological Constraints,''
\emph{Astrophys.\ J.} \textbf{938}, 110 (2022),
DOI: \href{https://doi.org/10.3847/1538-4357/ac8e04}{10.3847/1538-4357/ac8e04}.

\bibitem{SH0ES2022}
A.~G.~Riess \textit{et al.},
``A Comprehensive Measurement of the Local Value of the Hubble Constant with 1 km s$^{-1}$ Mpc$^{-1}$ Uncertainty from the Hubble Space Telescope and the SH0ES Team,''
\emph{Astrophys.\ J.\ Lett.} \textbf{934}, L7 (2022),
DOI: \href{https://doi.org/10.3847/2041-8213/ac5c5b}{10.3847/2041-8213/ac5c5b}.

\bibitem{BOSSDR12}
S.~Alam \textit{et al.},
``The clustering of galaxies in the completed SDSS-III Baryon Oscillation Spectroscopic Survey: cosmological analysis of the DR12 galaxy sample,''
\emph{Mon.\ Not.\ R.\ Astron.\ Soc.} \textbf{470}, 2617 (2017),
DOI: \href{https://doi.org/10.1093/mnras/stx721}{10.1093/mnras/stx721}.

\bibitem{eBOSSDR16}
S.~Alam \textit{et al.},
``Completed SDSS-IV extended Baryon Oscillation Spectroscopic Survey: Cosmological implications from two decades of spectroscopic surveys at the Apache Point Observatory,''
\emph{Phys.\ Rev.\ D} \textbf{103}, 083533 (2021),
DOI: \href{https://doi.org/10.1103/PhysRevD.103.083533}{10.1103/PhysRevD.103.083533}.

\bibitem{DESI2024VI}
DESI Collaboration,
``DESI 2024 VI: cosmological constraints from the measurements of baryon acoustic oscillations,''
\emph{J.\ Cosmol.\ Astropart.\ Phys.} \textbf{02} (2025) 021,
DOI: \href{https://doi.org/10.1088/1475-7516/2025/02/021}{10.1088/1475-7516/2025/02/021}.

\bibitem{Moresco2012}
M.~Moresco \textit{et al.},
``Improved constraints on the expansion rate of the Universe up to $z\sim1.1$ from the spectroscopic evolution of cosmic chronometers,''
\emph{J.\ Cosmol.\ Astropart.\ Phys.} \textbf{08} (2012) 006,
DOI: \href{https://doi.org/10.1088/1475-7516/2012/08/006}{10.1088/1475-7516/2012/08/006}.

\bibitem{Simon2005}
J.~Simon, L.~Verde, and R.~Jimenez,
``Constraints on the redshift dependence of the dark energy potential,''
\emph{Phys.\ Rev.\ D} \textbf{71}, 123001 (2005),
DOI: \href{https://doi.org/10.1103/PhysRevD.71.123001}{10.1103/PhysRevD.71.123001}.

\bibitem{Stern2010}
D.~Stern \textit{et al.},
``Cosmic chronometers: constraining the equation of state of dark energy. I: $H(z)$ measurements,''
\emph{J.\ Cosmol.\ Astropart.\ Phys.} \textbf{02} (2010) 008,
DOI: \href{https://doi.org/10.1088/1475-7516/2010/02/008}{10.1088/1475-7516/2010/02/008}.

\bibitem{Planck2018VI}
N.~Aghanim \textit{et al.} (Planck Collaboration),
``Planck 2018 results. VI. Cosmological parameters,''
\emph{Astron.\ Astrophys.} \textbf{641}, A6 (2020),
DOI: \href{https://doi.org/10.1051/0004-6361/201833910}{10.1051/0004-6361/201833910}.

\bibitem{Planck2018VIII}
N.~Aghanim \textit{et al.} (Planck Collaboration),
``Planck 2018 results. VIII. Gravitational lensing,''
\emph{Astron.\ Astrophys.} \textbf{641}, A8 (2020),
DOI: \href{https://doi.org/10.1051/0004-6361/201833886}{10.1051/0004-6361/201833886}.

\bibitem{Belgacem2018}
E.~Belgacem, Y.~Dirian, S.~Foffa, and M.~Maggiore,
``Modified gravitational-wave propagation and standard sirens,''
\emph{Phys.\ Rev.\ D} \textbf{98}, 023510 (2018),
DOI: \href{https://doi.org/10.1103/PhysRevD.98.023510}{10.1103/PhysRevD.98.023510}.

\bibitem{Nishizawa2018}
A.~Nishizawa,
``Generalized framework for testing gravity with gravitational-wave propagation,''
\emph{Phys.\ Rev.\ D} \textbf{97}, 104037 (2018),
DOI: \href{https://doi.org/10.1103/PhysRevD.97.104037}{10.1103/PhysRevD.97.104037}.

\end{thebibliography}

\end{document}
