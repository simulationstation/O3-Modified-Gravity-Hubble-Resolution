\documentclass[reprint,aps,prd,nofootinbib,floatfix]{revtex4-2}

% ==================== Packages ====================
\usepackage[T1]{fontenc}
\usepackage{lmodern}
\usepackage{amsmath,amssymb,mathtools}
\usepackage{graphicx}
\usepackage{booktabs}
\usepackage[hidelinks]{hyperref}

\begin{document}

\title{Cosmological Implications of the GWTC-3 Modified-Propagation Anomaly: Early-Anchor Recalibration, CMB-Lensing Refit, and Early-Universe H0 Inference Bias}
\author{Aiden B. Smith}
\affiliation{Independent Researcher}
\date{February 10, 2026}

\begin{abstract}
We project the GWTC-3 O3 modified-gravity propagation posterior into late- and early-universe inference tests using an updated Planck-facing calibration chain. The upstream O3 anomaly replication repository is archived on Zenodo (DOI: \href{https://doi.org/10.5281/zenodo.18585598}{10.5281/zenodo.18585598}). A 60-restart Planck+MG global refit defines an updated early anchor, $H_0^{\mathrm{Planck,MG}}=68.005$ (p50), $\Omega_m^{\mathrm{Planck,MG}}=0.30643$ (p50), and $A_{\mathrm{lens}}=1.0428$ (p50).

After rebasing late-time transfer sweeps to this anchor, the constrained anchor-relief posterior is $\mathcal{R}_{\mathrm{anchor}}^{\mathrm{GR}}=0.1545$ (mean; p16/p50/p84 $=0.108/0.147/0.189$). A 10-case robustness grid gives posterior-shift relief mean $0.5296$ (p50 $0.5125$), while a joint SN+BAO+CC transfer model yields $\mathcal{R}_{\mathrm{joint}}=0.8329$ (mean) but no Bayes preference for explicit transfer terms ($\log \mathrm{BF}=-0.533$).

For CMB lensing, direct CAMB propagation of MG draws still predicts baseline suppression at $L\!\sim\!100$ and $L\!\sim\!300$ (medians $-15.29\%$ and $-9.49\%$). However, an MG-aware lensing refit recovers good fits (median $\chi^2=8.06$ vs Planck-reference $9.04$; 100\% of refit draws better than reference) with median $M_\star^2(z\!=\!0)/M_\star^2(z\!\gg\!1)\simeq0.901$. Compressed-$\theta_\star$ inversions under GR assumptions bias inferred $H_0$ upward (mean $\Delta H_0=+1.88$ km s$^{-1}$ Mpc$^{-1}$ for fixed $\Omega_m$, or $+4.55$ for lensing-proxy $\Omega_m$), indicating that early-universe inference assumptions can materially shift recovered expansion rates in this MG-conditioned scenario.
\end{abstract}
\maketitle

\section{Scope and Framing}
This work treats the O3 modified-propagation signal phenomenologically: given the inferred posterior, what cosmological consequences follow? The O3 anomaly analysis and data products are archived on Zenodo~\cite{O3Zenodo}. We do not re-argue detection significance in this manuscript.

Modified GW propagation has been explored in theory-forward frameworks~\cite{Belgacem2018,Nishizawa2018}. Here, we use a data-driven posterior and update the pipeline to answer three questions in one chain:
\begin{enumerate}
\item How much late-time Hubble tension relief remains after recalibrating the early anchor?
\item Does Planck 2018 lensing necessarily reject this posterior, or can an MG-aware refit absorb the suppression?
\item How much can GR-based early-universe compression bias inferred $H_0$ if MG truth is assumed?
\end{enumerate}

\section{Pipeline Summary}
Posterior draws are taken from
\texttt{outputs/finalization/highpower\_multistart\_v2/M0\_start101}
and propagated through four linked stages:
\begin{enumerate}
\item \textbf{Global Planck+MG recalibration:} 60-restart multistart fit (\texttt{cpuset 0-59}) to establish updated early-anchor reference values.
\item \textbf{Late-time rebasing:} constrained/pilot transfer sweeps are rebased to the new Planck-like anchor and recompressed into a final relief posterior.
\item \textbf{CMB lensing forecasts:} baseline draw-level CAMB projection to Planck 2018 lensing bandpowers, followed by an MG-aware two-parameter lensing refit.
\item \textbf{Compressed early-universe inversion:} GR inversion of $\theta_\star=r_d/D_M(z_\star)$ under fixed-$\Omega_m$ and lensing-proxy-$\Omega_m$ assumptions.
\end{enumerate}
These are targeted forecasts and refits, not a full MG TT/TE/EE perturbation-sector likelihood analysis.

\section{Results}

\subsection{Updated early anchor from the global Planck+MG fit}
The 60-restart Planck+MG run completed all restarts with 5 converged minima and 55 max-evaluation exits. Using converged minima only, we obtain:
\begin{equation}
H_0^{\mathrm{Planck,MG}}=68.005302\;(\mathrm{p50}),\quad
\Omega_m^{\mathrm{Planck,MG}}=0.30643039\;(\mathrm{p50}),\quad
A_{\mathrm{lens}}=1.0428117\;(\mathrm{p50}).
\end{equation}

With local reference $H_0^{\mathrm{local}}=73.0$, the baseline gap used in rebased relief calculations is
\begin{equation}
\Delta H_0^{\mathrm{base}}=\left|H_0^{\mathrm{local}}-H_0^{\mathrm{Planck,MG}}\right|=4.994698.
\end{equation}

\subsection{Late-time relief after rebasing}
After rebasing constrained transfer sweeps to the updated early anchor and applying Monte Carlo calibration:
\begin{equation}
\mathcal{R}_{\mathrm{anchor}}^{\mathrm{GR}} = 0.1545
\quad\text{(mean; p16/p50/p84 }=0.1085/0.1475/0.1891\text{)}.
\end{equation}

Independent robustness and joint-fit diagnostics are:
\begin{itemize}
\item 10-case robustness grid: posterior-shift relief mean $0.5296$ (p50 $0.5125$, p84 $0.5453$), with zero failed cases.
\item Joint SN+BAO+CC transfer fit: relief posterior mean $0.8329$ (p50 $0.8386$), but
\begin{equation}
\log\mathrm{BF}_{\mathrm{transfer/no\text{-}transfer}}=-0.533,
\end{equation}
so explicit transfer terms are not favored in this setup.
\end{itemize}

\begin{figure}[htbp]
\centering
\includegraphics[width=0.95\linewidth]{hubble_tension_assets/relief_vs_highz_bias_rebased.png}
\caption{Rebased relief sensitivity to high-$z$ transfer bias from constrained and pilot sweeps. The final calibrated posterior is centered at $\mathcal{R}_{\mathrm{anchor}}^{\mathrm{GR}}\approx0.15$.}
\label{fig:relief_rebased}
\end{figure}

\subsection{CMB lensing: baseline suppression and MG-aware refit}
Baseline draw-level CAMB projection against Planck 2018 lensing bandpowers (\texttt{consext8}, 64 draws) gives:
\begin{equation}
\left.\frac{C_L^{\phi\phi}(\mathrm{MG})}{C_L^{\phi\phi}(\mathrm{Planck\ ref})}\right|_{L\approx106}
=0.847^{+0.091}_{-0.127},
\end{equation}
\begin{equation}
\left.\frac{C_L^{\phi\phi}(\mathrm{MG})}{C_L^{\phi\phi}(\mathrm{Planck\ ref})}\right|_{L\approx286}
=0.905^{+0.068}_{-0.080},
\end{equation}
with median suppressions of $-15.29\%$ and $-9.49\%$. The baseline fit quality is poor relative to the Planck-reference model:
\begin{equation}
\chi^2_{\mathrm{MG,baseline}}\;(\mathrm{median})=51.77,\qquad
\chi^2_{\mathrm{Planck\ ref}}=9.04,
\end{equation}
and only $3.1\%$ of draws outperform the reference. A 32-draw cross-check from an independent posterior sample is more discrepant ($-18.66\%$ at $L\approx106$, $-11.29\%$ at $L\approx286$; $p_{\mathrm{better}}=0$).

We then perform an MG-aware lensing refit (32 draws) with a phenomenological effective-$M_\star^2$ amplitude plus $\ell$-tilt response. This removes the baseline mismatch:
\begin{equation}
\chi^2_{\mathrm{MG\ refit}}\;(\mathrm{median})=8.06,
\end{equation}
better than the Planck-reference $\chi^2=9.04$ in 100\% of refit draws. The fitted median response corresponds to
\begin{equation}
\frac{M_\star^2(z=0)}{M_\star^2(z\gg1)}\simeq0.901
\end{equation}
(about a 9.9\% drop), with small residual suppression at $L\approx286$.

\begin{figure}[htbp]
\centering
\includegraphics[width=0.95\linewidth]{hubble_tension_assets/clpp_overlay_baseline_vs_mgfit.png}
\caption{Planck 2018 lensing bandpowers with baseline MG projection and MG-aware refit overlay. The refit absorbs the baseline suppression and restores near-reference fit quality.}
\label{fig:clpp_refit}
\end{figure}

\subsection{Early-universe GR inversion biases inferred $H_0$ upward}
Using the rebased early-anchor assumptions in compressed-$\theta_\star$ inversion:
\begin{itemize}
\item fixed $\Omega_m=\Omega_m^{\mathrm{Planck,MG}}$:
$H_{0,\mathrm{inferred}}$ mean $72.394$ (p50 $73.170$), with mean $\Delta H_0=+1.876$ km s$^{-1}$ Mpc$^{-1}$ relative to draw-level truth;
\item lensing-proxy $\Omega_m$:
$H_{0,\mathrm{inferred}}$ mean $75.065$ (p50 $75.226$), with mean $\Delta H_0=+4.547$ km s$^{-1}$ Mpc$^{-1}$.
\end{itemize}

Relative to the recalibrated early anchor $H_0^{\mathrm{Planck,MG}}=68.005$, the posterior medians shift by approximately:
\begin{equation}
\Delta H_0^{\mathrm{truth}}\approx+2.39,\quad
\Delta H_0^{\mathrm{fixed\ inversion}}\approx+5.16,\quad
\Delta H_0^{\mathrm{lensing\ inversion}}\approx+7.22\;\mathrm{km\ s^{-1}\ Mpc^{-1}}.
\end{equation}

\begin{figure}[htbp]
\centering
\begin{minipage}{0.49\linewidth}
\centering
\includegraphics[width=\linewidth]{hubble_tension_assets/h0_true_vs_inferred_hist_fixed_rebased.png}
\end{minipage}
\hfill
\begin{minipage}{0.49\linewidth}
\centering
\includegraphics[width=\linewidth]{hubble_tension_assets/h0_true_vs_inferred_hist_lensing_rebased.png}
\end{minipage}
\caption{Draw-level $H_0$ truth versus GR-inferred $H_0$ under compressed-$\theta_\star$ inversion with fixed-$\Omega_m$ (left) and lensing-proxy-$\Omega_m$ (right). Both assumptions bias inferred $H_0$ upward, with larger shift in the lensing-proxy case.}
\label{fig:h0_bias}
\end{figure}

\section{Interpretation}
The updated pipeline supports three conclusions.

First, recalibrating the Planck-like early anchor materially lowers the constrained anchor-relief estimate (now centered near $\sim0.15$), while robustness-grid and joint-transfer diagnostics still indicate that meaningful upward $H_0$ shifts are available in broader transfer-conditioned settings.

Second, baseline CAMB propagation of the O3 MG posterior predicts lensing suppression, but this does not by itself imply exclusion once MG-aware response freedom is included: the phenomenological refit reaches $\chi^2$ comparable to or better than the Planck-reference model.

Third, if MG truth holds while early-universe compression is interpreted under GR assumptions, inferred $H_0$ can be biased high by order $+2$ to $+5$ km s$^{-1}$ Mpc$^{-1}$ relative to posterior truth, and by up to $\sim+7$ km s$^{-1}$ Mpc$^{-1}$ relative to the recalibrated Planck-like anchor median.

Within this study scope, the largest systematic lever is not only late-time transfer, but also model-assumption bias in early-universe inference compression.

\clearpage

\section*{Reproducibility}
Core scripts used in this follow-up are:
\begin{itemize}
\item \texttt{scripts/run\_planck\_global\_mg\_refit\_multistart.py}
\item \texttt{scripts/rebase\_bias\_transfer\_sweep\_to\_planck\_ref.py}
\item \texttt{scripts/run\_hubble\_tension\_final\_relief\_posterior.py}
\item \texttt{scripts/run\_hubble\_tension\_mg\_forecast\_robustness\_grid.py}
\item \texttt{scripts/run\_joint\_transfer\_bias\_fit.py}
\item \texttt{scripts/run\_hubble\_tension\_cmb\_forecast.py}
\item \texttt{scripts/run\_hubble\_tension\_mg\_lensing\_refit.py}
\item \texttt{scripts/run\_hubble\_tension\_early\_universe\_bias.py}
\end{itemize}

\acknowledgments
This work relied extensively on A.I.-assisted tools for code development, pipeline execution support, figure generation, and manuscript drafting/editing.

\section*{Data Availability and DOIs}
The follow-up uses posterior products from the O3 anomaly pipeline and public cosmology datasets. Data provenance and DOIs are:
\begin{itemize}
\item O3 modified-gravity tension anomaly repository (Zenodo): DOI \href{https://doi.org/10.5281/zenodo.18585598}{10.5281/zenodo.18585598}.
\item O3 search-sensitivity injection data used in upstream calibration (Zenodo): DOI \href{https://doi.org/10.5281/zenodo.7890437}{10.5281/zenodo.7890437}.
\item GWTC-3 catalog paper: DOI \href{https://doi.org/10.1103/PhysRevX.13.041039}{10.1103/PhysRevX.13.041039}.
\item Pantheon+ cosmology constraints: DOI \href{https://doi.org/10.3847/1538-4357/ac8e04}{10.3847/1538-4357/ac8e04}.
\item SH0ES local-$H_0$ reference: DOI \href{https://doi.org/10.3847/2041-8213/ac5c5b}{10.3847/2041-8213/ac5c5b}.
\item SDSS DR12 BOSS consensus BAO (source of \texttt{sdss\_DR12Consensus\_bao.dat}): DOI \href{https://doi.org/10.1093/mnras/stx721}{10.1093/mnras/stx721}.
\item eBOSS DR16 cosmological compilation (source class for \texttt{sdss\_DR16\_LRG\_BAO\_DMDH.dat}): DOI \href{https://doi.org/10.1103/PhysRevD.103.083533}{10.1103/PhysRevD.103.083533}.
\item DESI 2024 BAO cosmological constraints (source class for \texttt{desi\_2024\_gaussian\_bao\_ALL\_GCcomb\_mean.txt}): DOI \href{https://doi.org/10.1088/1475-7516/2025/02/021}{10.1088/1475-7516/2025/02/021}.
\item Cosmic-chronometer compilation components used in \texttt{Hz\_BC03\_all.dat}: DOIs \href{https://doi.org/10.1088/1475-7516/2012/08/006}{10.1088/1475-7516/2012/08/006}, \href{https://doi.org/10.1103/PhysRevD.71.123001}{10.1103/PhysRevD.71.123001}, and \href{https://doi.org/10.1088/1475-7516/2010/02/008}{10.1088/1475-7516/2010/02/008}.
\item Planck 2018 cosmological-parameter and lensing references: DOIs \href{https://doi.org/10.1051/0004-6361/201833910}{10.1051/0004-6361/201833910} and \href{https://doi.org/10.1051/0004-6361/201833886}{10.1051/0004-6361/201833886}.
\end{itemize}

\begin{thebibliography}{99}

\bibitem{O3Zenodo}
A.~B.~Smith,
``O3 Modified Gravity Tension Replication,''
Zenodo (2026), DOI: \href{https://doi.org/10.5281/zenodo.18585598}{10.5281/zenodo.18585598}.

\bibitem{O3InjZenodo}
LIGO Scientific Collaboration, Virgo Collaboration, and KAGRA Collaboration,
``GWTC-3: Compact Binary Coalescences Observed by LIGO and Virgo During the Second Part of the Third Observing Run --- O3 search sensitivity estimates,''
Zenodo (2023), DOI: \href{https://doi.org/10.5281/zenodo.7890437}{10.5281/zenodo.7890437}.

\bibitem{GWTC3}
R.~Abbott \textit{et al.} (LIGO Scientific Collaboration, Virgo Collaboration, and KAGRA Collaboration),
``GWTC-3: Compact Binary Coalescences Observed by LIGO and Virgo During the Second Part of the Third Observing Run,''
\emph{Phys.\ Rev.\ X} \textbf{13}, 041039 (2023),
DOI: \href{https://doi.org/10.1103/PhysRevX.13.041039}{10.1103/PhysRevX.13.041039}.

\bibitem{PantheonPlus}
D.~Brout \textit{et al.},
``The Pantheon+ Analysis: Cosmological Constraints,''
\emph{Astrophys.\ J.} \textbf{938}, 110 (2022),
DOI: \href{https://doi.org/10.3847/1538-4357/ac8e04}{10.3847/1538-4357/ac8e04}.

\bibitem{SH0ES2022}
A.~G.~Riess \textit{et al.},
``A Comprehensive Measurement of the Local Value of the Hubble Constant with 1 km s$^{-1}$ Mpc$^{-1}$ Uncertainty from the Hubble Space Telescope and the SH0ES Team,''
\emph{Astrophys.\ J.\ Lett.} \textbf{934}, L7 (2022),
DOI: \href{https://doi.org/10.3847/2041-8213/ac5c5b}{10.3847/2041-8213/ac5c5b}.

\bibitem{BOSSDR12}
S.~Alam \textit{et al.},
``The clustering of galaxies in the completed SDSS-III Baryon Oscillation Spectroscopic Survey: cosmological analysis of the DR12 galaxy sample,''
\emph{Mon.\ Not.\ R.\ Astron.\ Soc.} \textbf{470}, 2617 (2017),
DOI: \href{https://doi.org/10.1093/mnras/stx721}{10.1093/mnras/stx721}.

\bibitem{eBOSSDR16}
S.~Alam \textit{et al.},
``Completed SDSS-IV extended Baryon Oscillation Spectroscopic Survey: Cosmological implications from two decades of spectroscopic surveys at the Apache Point Observatory,''
\emph{Phys.\ Rev.\ D} \textbf{103}, 083533 (2021),
DOI: \href{https://doi.org/10.1103/PhysRevD.103.083533}{10.1103/PhysRevD.103.083533}.

\bibitem{DESI2024VI}
DESI Collaboration,
``DESI 2024 VI: cosmological constraints from the measurements of baryon acoustic oscillations,''
\emph{J.\ Cosmol.\ Astropart.\ Phys.} \textbf{02} (2025) 021,
DOI: \href{https://doi.org/10.1088/1475-7516/2025/02/021}{10.1088/1475-7516/2025/02/021}.

\bibitem{Moresco2012}
M.~Moresco \textit{et al.},
``Improved constraints on the expansion rate of the Universe up to $z\sim1.1$ from the spectroscopic evolution of cosmic chronometers,''
\emph{J.\ Cosmol.\ Astropart.\ Phys.} \textbf{08} (2012) 006,
DOI: \href{https://doi.org/10.1088/1475-7516/2012/08/006}{10.1088/1475-7516/2012/08/006}.

\bibitem{Simon2005}
J.~Simon, L.~Verde, and R.~Jimenez,
``Constraints on the redshift dependence of the dark energy potential,''
\emph{Phys.\ Rev.\ D} \textbf{71}, 123001 (2005),
DOI: \href{https://doi.org/10.1103/PhysRevD.71.123001}{10.1103/PhysRevD.71.123001}.

\bibitem{Stern2010}
D.~Stern \textit{et al.},
``Cosmic chronometers: constraining the equation of state of dark energy. I: $H(z)$ measurements,''
\emph{J.\ Cosmol.\ Astropart.\ Phys.} \textbf{02} (2010) 008,
DOI: \href{https://doi.org/10.1088/1475-7516/2010/02/008}{10.1088/1475-7516/2010/02/008}.

\bibitem{Planck2018VI}
N.~Aghanim \textit{et al.} (Planck Collaboration),
``Planck 2018 results. VI. Cosmological parameters,''
\emph{Astron.\ Astrophys.} \textbf{641}, A6 (2020),
DOI: \href{https://doi.org/10.1051/0004-6361/201833910}{10.1051/0004-6361/201833910}.

\bibitem{Planck2018VIII}
N.~Aghanim \textit{et al.} (Planck Collaboration),
``Planck 2018 results. VIII. Gravitational lensing,''
\emph{Astron.\ Astrophys.} \textbf{641}, A8 (2020),
DOI: \href{https://doi.org/10.1051/0004-6361/201833886}{10.1051/0004-6361/201833886}.

\bibitem{Belgacem2018}
E.~Belgacem, Y.~Dirian, S.~Foffa, and M.~Maggiore,
``Modified gravitational-wave propagation and standard sirens,''
\emph{Phys.\ Rev.\ D} \textbf{98}, 023510 (2018),
DOI: \href{https://doi.org/10.1103/PhysRevD.98.023510}{10.1103/PhysRevD.98.023510}.

\bibitem{Nishizawa2018}
A.~Nishizawa,
``Generalized framework for testing gravity with gravitational-wave propagation,''
\emph{Phys.\ Rev.\ D} \textbf{97}, 104037 (2018),
DOI: \href{https://doi.org/10.1103/PhysRevD.97.104037}{10.1103/PhysRevD.97.104037}.

\end{thebibliography}

\end{document}
